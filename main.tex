\documentclass[12pt,a4paper]{article}

\usepackage[utf8]{inputenc}
\usepackage[english,russian]{babel}
\usepackage{indentfirst}
\usepackage{misccorr}
\usepackage{graphicx}
\usepackage{amsmath}
\usepackage{amsthm}
\usepackage{enumitem}
\usepackage{hyperref}

\hypersetup{
    colorlinks=true,
    linkcolor=blue,
    filecolor=magenta,      
    urlcolor=cyan,
}

\title{Интегралы}
\author{\href{https://t.me/stewkk}{@stewkk}}

\newtheorem{theorem}{Th}
\newtheorem*{remark}{Примечание}

\begin{document}
\maketitle
Конспект будет дополняться, актуальная версия на
\href{https://github.com/stewkk/integrals}{GitHub}.
\section{Неопределенные интегралы}
\subsection{Первообразные функции и неопределенные интегралы}
Задачей дифференциорования являлось нахождение для функции ее производной.
Поставим обратную задачу.
Для каждой функции $f(x)$ найти $F(x)$, $F'(x)=f(x)$
    \begin{gather*}
    f(x) = x^5\\
    F(x) = \frac{x^6}{6} + 5\\
    F'(x) = f(x)
    \end{gather*}
Функция $F(x)$ является первообразной для $f(x)$, если $F'(x)=f(x)$ или $dF(x)=f(x)dx$.
Т.е. первообразная для любой функции $f(x)$ существует с точностью до постоянного слагаемого.

\begin{theorem}
    если $F_1(x)$ и $F_2(x)$ являются первообразными для $f(x)$, то они отличаются на постоянное слагаемое.
\end{theorem}

\begin{proof}
\begin{gather*}
    f(x),\ F_1(x),\ F_2(x) \\
    \Phi(x) = F_1(x) - F_2(x) \\
    \Phi'(x) = (F_1(x) - F_2(x))' \\
    \Phi'(x) = F_1'(x) - F_2'(x) \\
    \Phi'(x) = f(x) - f(x) \\
    \Phi'(x) = 0 \\
    (F_1(x) - F_2(x))' = 0 \\
    \Phi(x) = const \\
    \Phi(x) = C \\
    C = F_1(x) - F_2(x) \\
    F_1(x) = C + F_2(x)
\end{gather*}
\end{proof}

Определение. если мы рассмотрим множество всех первообразных, то можно вести речь о неопределенном интеграле:
\[ \int f(x)dx = F(x) + C \]

Свойства неопределенного интеграла:
\begin{enumerate}
\item{производная от неопределенного интеграла равна подынтегральной функции, а дифференциал неопределенного интеграла равен подынтегральному выражению:
\[ \Big(\int f(x)dx \Big)' = \big( F(x) + C \big)' \]
\[ \Big( \int f(x)dx \Big)' = F'(x) + C' \]
\[ \Big( \int f(x)dx \Big)' = f(x) + 0 \]
}
\item{
Постоянный множитель можно выносить за знак интеграла:
\[ \int kf(x)dx = k \int f(x)dx \]
\[ \Big(\int kf(x)dx \Big)' = \Big(k \int f(x)dx \Big)' = k \Big( \int f(x)dx \Big)' = kf(x) \]
}
\item{
Неопределенный интеграл суммы или разности нескольких выражений равен сумме или разности интегралов этих выражений:
\[ \int \big(f(x) \pm g(x) \big)dx = \int f(x)dx \pm \int g(x)dx \]
}
\[ \Big(\int \big(f(x) \pm g(x) \big)dx \Big)' = \Big(\int f(x)dx \pm \int g(x)dx \Big)'  \]
\[ f(x) \pm g(x) = \Big(\int f(x)dx \Big)' \pm \Big(\int g(x)dx \Big)' \]
\[ f(x) \pm g(x) = f(x) \pm g(x) \]
\end{enumerate}
\subsection{Таблица основных неопределенных интегралов}

\begin{enumerate}[leftmargin=*, itemsep=0.4ex, before={\everymath{\displaystyle}}]%
\item{
$\int kdx = kx + C$
}
\item{
$\int x^ndx = \frac{x^{n + 1}}{n + 1} + C$
}
\item{
$\int \sin xdx = - \cos x + C$
}
\item{
$\int \cos xdx = \sin x + C$
}
\item{
$\int \frac{1}{\cos^2 x}dx = \tg x + C$
}
\item{
$\int - \frac{1}{\sin^2 x} dx = \ctg x + C$
}
\item{
$\int e^x dx = e^x + C$
}
\item{
\label{eq-8}
$\int \frac{1}{x}dx = \ln |x| + C$
}
\item{
$\int a^xdx = \frac{a^x}{\ln a} + C$
}
\item{
$\int \frac{dx}{\sqrt{1 - x^2}} = \arcsin x + C$
}
\item{
$\int \frac{dx}{1 + x^2} = \arctg x + C$
}
\end{enumerate}
 Пример вывода(\ref{eq-8}):
\begin{proof}
\[ \int (\log_{a} x)'dx = \int \frac{1}{x \ln a}dx\]
\[ \log_{a} x = \frac {1}{\ln a} \int \frac{1}{x} dx \]
\[ \log_{a} x * \ln a = \int \frac{1}{x} dx \]
\end{proof}

\subsection{Примеры}
\begin{enumerate}[label=(\roman*), leftmargin=*, itemsep=0.4ex, before={\everymath{\displaystyle}}]%
\item{
$\int \sqrt[3]{x} dx = \frac{x^{\frac{4}{3}}}{\frac{4}{3}} + C$
}
\item{
$
\int (7x^6 + \frac{3}{x^2})dx = 7 \int x^6 dx + 3 \int x^{-2} dx
= 7 * \frac{x^7}{7} + 3 * \frac{x^{-1}}{-1} + C =\\ = x^7 - \frac{3}{x} + C
$
}
\item{
$
\int (\sqrt{2x} + \sqrt{\frac{2}{x}})dx = \sqrt{2} \int x^{1/2}dx +
\sqrt{2} \int x^{-1/2}dx =\\ = \frac{\sqrt{2} x^{3/2}}{3/2} +
\frac{\sqrt{2} x^{1/2}}{1/2} + C
$
}
\item{
$\int \frac{dx}{\cos 2x + \sin^2 x} =
\int \frac{dx}{\cos^2 x - \sin^2 x + \sin^2 x} =
\int \frac{dx}{cos^2 x} =
\tg x + C$
}
\item{
$
\int e^{6 - 2x} dx =
\int \frac{e^{6 - 2x}d(6 - 2x)}{-2} =
-\frac{1}{2} e^{6 - 2x} + C
$
}
\end{enumerate}
\clearpage
\section{Методы вычисления интегралов}
\subsection{Замена или подведение под дифференциал}
\subsubsection{замена}
\begin{align*}
\int 3^{\frac{x}{4}} dx = \Bigg|
\begin{split}
\frac{x}{4} & = t \\
x & = 4t \\
dx & = 4dt
\end{split}
\Bigg| = \int 3^t 4dt = 4 \int 3^t dt = 4 * \frac{3^t}{\ln 3} + C =
\frac{4 * 3^{\frac{x}{4}}}{\ln 3} + C
\end{align*}
Под знаком интеграла необходимо увидеть функцию и ее производную
с точностью до постоянного множителя(слагаемого).
Тогда на данную функцию вводим замену.
\subsubsection{подведение под дифференциал}
\[
\int x^{\frac{x}{4}} dx = \int 4 * 3^{\frac{x}{4}} d\frac{x}{4}
= 4 \int 3^{\frac{x}{4}} d\frac{x}{4} =
4 * \frac{3^{\frac{x}{4}}}{\ln 3} + C
\]
\subsection{Метод интегрирования по частям}
Метод основан на интегрировании произведения двух функций:
\[
    d(uv) = udv + vdu
\]
\[
\int d(uv) = \int \big( udv + vdu \big)
\]
\[
uv = \int udv + \int vdu
\]
\begin{equation}
\label{eq-1}
\int udv = uv - \int vdu
\end{equation}
Метод интегрирования по частям не позволяет выразить интеграл за
одно применение, но позволяет перейти к более простому.

Пример:
\begin{align*}
\int x * e^{2x} dx = \Bigg|
\begin{split}
u & = x\ \ \ \ \ du = dx \\
dv & = e^{2x} dx \ \ v = \int e^{2x} dx =
\int \frac{e^{2x} 2x}{2} = \frac{1}{2} * \frac{e^{2x}}{ \ln e} =
\frac{1}{2}e^{2x}
\end{split}
\Bigg| =
\end{align*}
\begin{remark}
Промежуточные действия в замене обычно не пишут.
\end{remark}
\[
= x * \frac{1}{2} e^{2x} - \int \frac{1}{2} e^{2x} dx =
\frac{x e^{2x}}{2} - \frac{1}{2} \int \frac{e^{2x} d2x}{2} =
\frac{xe^{2x}}{2} - \frac{e^{2x}}{4} + C
\]
\begin{remark}
"$C$"\ в промежуточных действиях обычно не пишут.
\end{remark}

Пример 2:
\begin{align*}
\int x^2 \ln x dx = \Bigg|
\begin{split}
u & = \ln x \ \ \ du = \frac{1}{x}dx\\
dv & = x^2 dx \ \ \ v = \frac{x^3}{3}
\end{split}
\Bigg| =
\ln x * \frac{x^3}{3} - \int \frac{x^3}{3} * \frac{1}{x} dx =
\end{align*}
\[
= \frac{x^3}{3} \ln x - \frac{1}{3} \int x^2 dx = \frac{x^3}{3} \ln x
= \frac{x^3}{3} \ln x - \frac{x^3}{9} + C = \frac{x^3}{3}
(\ln x - \frac{1}{3}) + C
\]

Формула (\ref{eq-1}) называется формулой интегрирования по частям.
Подынтегральное выражние в формуле (\ref{eq-1}) представляет собой
произведение двух множителей: $u$ и $dv$ каждый из которых
подбирается так, чтобы от выражения $dv$ можно было получить
интеграл, а $du$ уменьшало степень выражения, или хотя бы ее не
увеличивала.
Метод интегрирования по частям применяется при нахождении интеграла
вида, где подытнтегральные функции имеют вид:

1 группа:
\begin{enumerate}[leftmargin=*, itemsep=0.4ex, before={\everymath{\displaystyle}}]%
\item{$P(x) * e^{kx} dx$}
\item{$P(x) * \sin kx\ dx$}
\item{$P(x) * \cos kx\ dx$}
\end{enumerate}

2 группа:
\begin{enumerate}[leftmargin=*, itemsep=0.4ex, before={\everymath{\displaystyle}}]%
\item{$P(x) * \ln x\ dx$}
\item{$P(x) * \arcsin x\ dx$}
\item{$P(x) * \arccos x\ dx$}
\item{$P(x) * \arctg x\ dx$}
\item{$P(x) * \arcctg x\ dx$}
\end{enumerate}

В которых $P(x)$ - это многочлен $n$-ой степени (не применять при
боьшом $n$).
Применяя (\ref{eq-1}) к интегралам 1-ой группы за $u$ принимают
$P(x)$, за $dv$ - остальное.
При многочленах высокой степени ($n$) - формула может
применяться $n$-кратное число раз.

В интегралах 2 группы за $u$ принимают $\ln x, \arccos x,...$;
за $dv$: $P(x) dx$.
\subsection{Интегрирование рациональных дробей}
Рациональной дробью называется дробь вида $\frac{P(x)}{Q(x)}$, где
$P(x)$ и $Q(x)$ многочлены и $Q(x) \neq 0$.
Рациональная дробь может быть неправильной, когда степень числителя
больше степени знаменателя; а правильной, когда степень
числителя меньше степени знаменателя.

Поэтому, интегрирование рациональных дробей сводится к
интегрированию только правильных дробей.

Все правильные дроби разложим на виды:
\begin{enumerate}[leftmargin=*, itemsep=0.9ex, before={\everymath{\displaystyle}}]%
\item{
$
\int \frac{dx}{x \pm a} = \int \frac{d(x \pm a)}{x \pm a} =
\ln |x \pm a| + C
$
}
\item{
$
\int \frac{dx}{x^2 - a^2} = (*)\\
\frac{1}{x^2 - a^2} = \frac{1}{x - a} * \frac{1}{x + a}\\
\frac{A}{x - a} + \frac{B}{x + a} =
\frac{A(x + a) + B(x - a)}{(x - a)(x + a)} =
\frac{x(A + B) + a(A - B)}{(x - a)(x + a)}\\
\begin{cases}
A + B = 0\\
a(A - B) = 1
\end{cases}\\
\begin{cases}
A = -B\\
a(-2B) = 1
\end{cases}\\
B = -\frac{1}{2a}\\
A = \frac{1}{2a}\\
\frac{1}{x^2 + a^2} = \frac{1}{2a(x - a)} - \frac{1}{2a(x + a)}\\
(*) = \int \Big(\frac{dx}{2a(x - a)} = \frac{dx}{2a(x + a)} \Big) =
\frac{1}{2a} \int \frac{dx}{x - a} -
\frac{1}{2a} \int \frac{dx}{x + a} =\\
= \frac{1}{2a} (\ln |x - a| - \ln |x + a|) + C =
\frac{1}{2a} \ln \big| \frac{x - a}{x + a} \big| + C
$
}
\item{
$
\int \frac{dx}{(x \pm a)^2} = \int \frac{d |x \pm a|}{(x \pm a)^2} =
-(x \pm a)^{-1} + C
$
}
\item{
$
\int \frac{xdx}{(x \pm a)^2} =
\int \frac{x \pm a \mp a}{(x \pm a)^2} dx =
\int \frac{1}{x \pm a} d(x \pm a) \mp
\int \frac{a}{(x \pm a)^2} dx =\\
\ln |x \pm a| \mp a(\frac{-1}{x \pm a}) + C =
\ln |x \pm a| \pm \frac{a}{x \pm a} + C
$
}
\item{
$
\int \frac{xdx}{x^2 \pm a^2} =
\int \frac{xd(x^2 + a^2)}{x^2 \pm a^2} * \frac{1}{2x} =
\frac{1}{2} \int \frac{d(x^2 \pm a^2}{x^2 \pm a^2} =
\frac{1}{2} \ln |x^2 \pm a^2| + C
$
}
\item{
\begin{align*}
\int \frac{dx}{x^2 + a^2} = \Bigg|
\begin{split}
x & = at\\
dx & = adt
\end{split}
\Bigg|
= \int \frac{a\ dt}{a^2t^2 + a^2} =
\frac{1}{a} \int \frac{dt}{t^2 + 1} =
\frac{1}{a} \arctg \frac{x}{a} + C
\end{align*}
}
\end{enumerate}

Итого:
\begin{enumerate}[leftmargin=*, itemsep=0.4ex, before={\everymath{\displaystyle}}]%
\item{
$
\int \frac{dx}{x \pm a} =
\ln |x \pm a| + C
$
}
\item{
$
\int \frac{dx}{x^2 - a^2} =
\frac{1}{2a} \ln \big| \frac{x - a}{x + a} \big| + C
$
}
\item{
$
\int \frac{dx}{(x \pm a)^2} =
-(x \pm a)^{-1} + C
$
}
\item{
$
\int \frac{xdx}{(x \pm a)^2} =
\ln |x \pm a| \pm \frac{a}{x \pm a} + C
$
}
\item{
$
\int \frac{xdx}{x^2 \pm a^2} =
\frac{1}{2} \ln |x^2 \pm a^2| + C
$
}
\item{
$
\int \frac{dx}{x^2 + a^2} =
\frac{1}{a} \arctg \frac{x}{a} + C
$
}
\end{enumerate}
\subsubsection{примеры}
\begin{enumerate}[leftmargin=*, itemsep=0.9ex, before={\everymath{\displaystyle}}]%
\item{
$
t = x + 3\\
\int \frac{dx}{x^2 + 6x - 7} =
\int \frac{d(x + 3)}{(x + 3)^2 - 16} =
\int \frac{dt}{t^2 - 4^2} =
\frac{1}{8} \ln |\frac{t - 4}{t + 4}| + C =\\
= \frac{1}{8} \ln |\frac{x - 1}{x + 7}| + C\\
$
}
\item{
$
t = x - 3\\
\int \frac{6x + 1}{x^2 - 6x + 8} dx =
\int \frac{6x + 1}{(x - 3)^2 - 1^2} =
6\int \frac{(x - 3) + 3\frac{1}{6}}{(x - 3)^2 - 1^2} d(x - 3) =\\
= 6\int \frac{(x - 3)d(x - 3)}{(x - 3)^2 - 1} + 19 *
\int \frac{1}{x - 3^2 - 1^2}d(x - 3) =\\
= 6 \int \frac{t\ dt}{t^2 - 1} + 16 \int \frac{dt}{t^2 - 1}
$
}
\item{
$
\int \frac{x + 3}{x^2 - 8x + 25} dx =
\int \frac{x + 3}{(x - 3)^2 + 9} dx =
\int \frac{x - 4 + 7}{(x - 4)^2 + 9} dx
$
}
\end{enumerate}
\subsection{Интегрирование некоторых видов иррациональности}
\begin{enumerate}[leftmargin=*, itemsep=0.9ex, before={\everymath{\displaystyle}}]%
\item{
$
\int \frac{dx}{\sqrt{a^2 - x^2}} =
\begin{vmatrix} x = at \\ dx = adt \end{vmatrix} =
\int \frac{adt}{a^2 - a^2t^2} =
\int \frac{dt}{\sqrt{1 - t^2}} =\\
= \arcsin t + C = \arcsin \frac{x}{a} + C
$
}
\item{
$
\int \frac{dx}{\sqrt{x^2 \pm a}} =
\int \frac{dt}{t} = \ln |t| + C =
\ln |x + \sqrt{x^2 + a}| + C\\
t = x + \sqrt{x^2 \pm a^2}\\
dt = 1 + \frac{1 * 2x}{2 \sqrt{x^2 \pm a}} dx =
1 + \frac{x}{\sqrt{x^2 \pm a}} dx + C\\
= \int \frac{\sqrt{x^2 \pm a} + x}{\sqrt{x^2 \pm a}}dx\\
dt = \frac{t}{\sqrt{x^2 \pm a}} dx\\
\frac{dt}{t} = \frac{dx}{\sqrt{x^2 \pm a}}
$
}
\item{
$
\int \frac{dx}{\sqrt{x^2 + 5x + 6}} =
\int \frac{dx}{\sqrt{x^2 + 5x + (\frac{5}{2})^2 - (\frac{5}{2})^2 + 6}} =
\int \frac{dx}{\sqrt{(x - \frac{5}{2})^2 - \frac{1}{2^2}}} =\\
= \int \frac{d(x - \frac{5}{2})}{\sqrt{(x - \frac{5}{2})^2 - 
\frac{1}{4}}} =
\begin{vmatrix}
t = (x - \frac{5}{2}) + \sqrt{(x - \frac{5}{2})^2 - 
\frac{1}{4}}\\
dt = 1 + \frac{x - \frac{5}{2}}{\sqrt{(x - \frac{5}{2})^2
- \frac{1}{4}}}
\end{vmatrix}
= \sqrt{\frac{dt}{t}} =\\
= \ln |t| + C =
\ln \Big|(x - \frac{5}{2}) +
\sqrt{(x - \frac{5}{2})^2 - \frac{1}{4}} \Big| + C
$
}
\end{enumerate}
\subsection{Интегрирование тригонометрических функций}
\[
\int \sin mx * \cos mx\\
\int \cos mx * \cos mx\\
\int \sin mx * \sin mx
\]
Для вычисления интегралов данного вида используют тригонометрические
формулы данных тригонометрических функций.

Рассмотрим интегралы от тригонометрических функций содержащих степени
\subsubsection{нечетная + четная степень}
\[
\int \sin^m x \cos^n x dx
\]
при интегрировании таких функций используют основное тригонометрическое
тождество для выражения четной степени, а оставшуюся нечетную можно
представить в виде дифференциала.
Применяется если $m$ и $n$ - одна четная, другая нечетная.
\[
\int \cos^6 x \cos^3 x dx =
\int \sin^6 x * (1 - \sin^2 x) \cos x dx =
\]
\[
= \int (\sin^6 x - \sin^8 x)d \sin x =
\begin{vmatrix}
\sin x = t\\
\ \ 
\end{vmatrix}
= \int (t^6 - t^8)dt =
\]
\[
= \frac{t^7}{7} - \frac{t^9}{9} + C
= \frac{\sin^7 x}{7} - \frac{\sin^9 x}{9} + C
\]
\subsubsection{нечетная + нечетная}
\[
\int \sin^7 x \cos^3 x dx = \int \sin^7 x(1 - sin^2 x) *
\frac{\cos x\ d(\sin x)}{\cos x}
\]
\subsubsection{четная + четная}
выносим произодную функции в наименьшей из степеней
(можно воспользоваться формулой синуса двойного угла или
понижения степени)
\[
\int \sin^6 x \cos^2 x dx = \frac{1}{4} \int (2 \sin x \cos x)^2 *
\sin^4 x dx =
\]
\[
= \frac{1}{4} \int \sin^2 2x (\frac{1 - \cos 2x}{2})^2 dx =
\frac{1}{16} \int (1 - \cos^2 2x)(1 - \cos 2x)^2 dx =
\Bigg|... 
\]
\section{Определенные интегралы}
\end{document}
