\documentclass[12pt,a4paper]{article}

\usepackage[utf8]{inputenc}
\usepackage[english,russian]{babel}
\usepackage{indentfirst}
\usepackage{misccorr}
\usepackage{graphicx}
\usepackage{amsmath}
\usepackage{amsthm}
\usepackage{enumitem}

\title{Интегралы}
\author{@stewkk}

\newtheorem{theorem}{Th}

\begin{document}
\maketitle
\section{Неопределенные интегралы}
\subsection{Первообразные функции и неопределенные интегралы}
Задачей дифференциорования являлось нахождение для функции ее производной.
Поставим обратную задачу.
Для каждой функции $f(x)$ найти $F(x)$, $F'(x)=f(x)$
    \begin{gather*}
    f(x) = x^5\\
    F(x) = \frac{x^6}{6} + 5\\
    F'(x) = f(x)
    \end{gather*}
Функция $F(x)$ является первообразной для $f(x)$, если $F'(x)=f(x)$ или $dF(x)=f(x)dx$.
Т.е. первообразная для любой функции $f(x)$ существует с точностью до постоянного слагаемого.

\begin{theorem}
    если $F_1(x)$ и $F_2(x)$ являются первообразными для $f(x)$, то они отличаются на постоянное слагаемое.
\end{theorem}

\begin{proof}
\begin{gather*}
    f(x),\ F_1(x),\ F_2(x) \\
    \Phi(x) = F_1(x) - F_2(x) \\
    \Phi'(x) = (F_1(x) - F_2(x))' \\
    \Phi'(x) = F_1'(x) - F_2'(x) \\
    \Phi'(x) = f(x) - f(x) \\
    \Phi'(x) = 0 \\
    (F_1(x) - F_2(x))' = 0 \\
    \Phi(x) = const \\
    \Phi(x) = C \\
    C = F_1(x) - F_2(x) \\
    F_1(x) = C + F_2(x)
\end{gather*}
\end{proof}

Определение. если мы рассмотрим множество всех первообразных, то можно вести речь о неопределенном интеграле:
\[ \int f(x)dx = F(x) + C \]

Свойства неопределенного интеграла:
\begin{enumerate}
\item{производная от неопределенного интеграла равна подынтегральной функции, а дифференциал неопределенного интеграла равен подынтегральному выражению:
\[ (\int f(x)dx)' = (F(x) + C)' \]
\[ (\int f(x)dx)' = F'(x) + C' \]
\[ (\int f(x)dx)' = f(x) + 0 \]
}
\item{
Постоянный множитель можно выносить за знак интеграла:
\[ \int kf(x)dx = k \int f(x)dx \]
\[ (\int kf(x)dx)' = (k \int f(x)dx)' = k( \int f(x)dx)' = kf(x) \]
}
\item{
Неопределенный интеграл суммы или разности нескольких выражений равен сумме или разности интегралов этих выражений:
\[ \int (f(x) \pm g(x))dx = \int f(x)dx \pm \int g(x)dx \]
}
\[ (\int(f(x) \pm g(x))dx)' = (\int f(x)dx \pm \int g(x)dx )'  \]
\[ f(x) \pm g(x) = (\int f(x)dx)' \pm (\int g(x)dx)' \]
\[ f(x) \pm g(x) = f(x) \pm g(x) \]
\end{enumerate}
\subsection{Таблица основных неопределенных интегралов}

\begin{enumerate}[leftmargin=*, itemsep=0.4ex, before={\everymath{\displaystyle}}]%
\item{
$\int kdx = kx + C$
}
\item{
$\int x^ndx = \frac{x^{n + 1}}{n + 1} + C$
}
\item{
$\int \sin xdx = - \cos x + C$
}
\item{
$\int \cos xdx = \sin x + C$
}
\item{
$\int \frac{1}{\cos^2 x}dx = \tg x + C$
}
\item{
$\int - \frac{1}{\sin^2 x} dx = \ctg x + C$
}
\item{
$\int e^x dx = e^x + C$
}
\item{
$\int \frac{1}{x}dx = \ln |x| + C$
} \label{eq-8}
\item{
$\int a^xdx = \frac{a^x}{\ln a} + C$
}
\item{
$\int \frac{dx}{\sqrt{1 - x^2}} = \arcsin x + C$
}
\item{
$\int \frac{dx}{1 + x^2} = \arctg x + C$
}
\end{enumerate}
 Пример вывода(\ref{eq-8}):
\begin{proof}
\[ \int (\log_{a} x)'dx = \int \frac{1}{x \ln a}dx\]
\[ \log_{a} x = \frac {1}{\ln a} \int \frac{1}{x} dx \]
\[ \log_{a} x * \ln a = \int \frac{1}{x} dx \]
\end{proof}

\subsection{Примеры}
\begin{enumerate}[label=(\roman*), leftmargin=*, itemsep=0.4ex, before={\everymath{\displaystyle}}]%
\item{
$\int \sqrt[3]{x} dx = \frac{x^{\frac{4}{3}}}{\frac{4}{3}} + C$
}
\item{
$
\int (7x^6 + \frac{3}{x^2})dx = 7 \int x^6 dx + 3 \int x^{-2} dx
= 7 * \frac{x^7}{7} + 3 * \frac{x^{-1}}{-1} + C =\\ x^7 - \frac{3}{x} + C
$
}
\item{
$
\int (\sqrt{2x} + \sqrt{\frac{2}{x}})dx = \sqrt{2} \int x^{1/2}dx +
\sqrt{2} \int x^{-1/2}dx =\\ \frac{\sqrt{2} x^{3/2}}{3/2} +
\frac{\sqrt{2} x^{1/2}}{1/2} + C
$
}
\item{
$\int \frac{dx}{\cos 2x + \sin^2 x} =
\int \frac{dx}{\cos^2 x - \sin^2 x + \sin^2 x} =
\int \frac{dx}{cos^2 x} =
\tg x + C$
}
\item{
$
\int e^{6 - 2x} dx =
\int \frac{e^{6 - 2x}d(6 - 2x)}{-2} =
-\frac{1}{2} e^{6 - 2x} + C
$
}
\end{enumerate}
\end{document}
