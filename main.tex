\documentclass[12pt,a4paper]{article}

\usepackage[utf8]{inputenc}
\usepackage[english,russian]{babel}
\usepackage{indentfirst}
\usepackage{misccorr}
\usepackage{graphicx}
\usepackage{amsmath}
\usepackage{amsthm}

\title{Интегралы}
\author{@stewkk}

\newtheorem{theorem}{Th}

\begin{document}
\maketitle
\section{Неопределенные интегралы}
Задачей дифференциорования являлось нахождение для функции ее производной.
Поставим обратную задачу.
Для каждой функции $f(x)$ найти $F(x)$, $F'(x)=f(x)$
    \begin{gather*}
    f(x) = x^5\\
    F(x) = \frac{x^6}{6} + 5\\
    F'(x) = f(x)
    \end{gather*}
Функция $F(x)$ является первообразной для $f(x)$, если $F'(x)=f(x)$ или $dF(x)=f(x)dx$.
Т.е. первообразная для любой функции $f(x)$ существует с точностью до постоянного слагаемого.

\begin{theorem}
    если $F_1(x)$ и $F_2(x)$ являются первообразными для $f(x)$, то они отличаются на постоянное слагаемое.
\end{theorem}

\begin{proof}
\begin{gather*}
    f(x),\ F_1(x),\ F_2(x) \\
    \Phi(x) = F_1(x) - F_2(x) \\
    \Phi'(x) = (F_1(x) - F_2(x))' \\
    \Phi'(x) = F_1'(x) - F_2'(x) \\
    \Phi'(x) = f(x) - f(x) \\
    \Phi'(x) = 0 \\
    (F_1(x) - F_2(x))' = 0 \\
    \Phi(x) = const \\
    \Phi(x) = c \\
    c = F_1(x) - F_2(x) \\
    F_1(x) = c + F_2(x)
\end{gather*}
\end{proof}

Определение. если мы рассмотрим множество всех первообразных, то можно вести речь о неопределенном интеграле:
\[ \int f(x)dx = F(x) + c \]

Свойства неопределенного интеграла:
\begin{enumerate}
\item{производная от неопределенного интеграла равна подинтегральной функции, а дифференциал неопределенного интеграла равен подинтегральному выражению:
\[ (\int f(x)dx)' = (F(x) + c)' \]
\[ (\int f(x)dx)' = F'(x) + c' \]
\[ (\int f(x)dx)' = f(x) + 0 \]
}
\item{
Постоянный множитель можно выносить за знак интеграла:
\[ \int kf(x)dx = k \int f(x)dx \]
}

\end{enumerate}

\end{document}
